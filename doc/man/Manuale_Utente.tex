\documentclass[12pt]{scrreprt}

\usepackage{sty/preamble}

\title{Manuale Utente}
\subtitle{CLIMATE MONITORING}
\author{
	Iuri Antico \textit{matricola}:
	\texttt{753144}
	\and \\
	Michael Bernasconi \textit{matricola}:
	\texttt{752259}
	\and \\
	Gabriele Borgia \textit{matricola}:
	\texttt{753262}
	\and \\
	Beatrice Balzarini \textit{matricola}:
	\texttt{752257}
}
\date{\today}

\begin{document}

	\maketitle

	\tableofcontents
	\listoffigures
	\listoftables

	\chapter{Il Programma}
	\section{Introduzione}
		\textit{Climate Monitoring} è un'applicazione di monitoraggio di parametri climatici fornita da centri di monitoraggio sul territorio italiano, in grado di rendere disponibili, a operatori ambientali e comuni cittadini, i dati relativi alla propria zona di interesse.

		\subsection{Funzionamento generale dell'applicazione}
		I parametri climatici che possono essere rilevati sono:
		\begin{itemize}
		\item \textit{Vento}, la cui velocità è espressa in km/h;
		\item \textit{Umidità}, misurata in percercentuale (\%);
		\item \textit{Pressione}, misurata in hPa;
		\item \textit{Temperatura}, misurata in C°;
		\item \textit{Precipitazione}, misurata in mm di pioggia;
		\item \textit{Altitudine dei ghiacciai}, misurata in m;
		\item \textit{Massa dei ghiacciai}, misurata in kg.
		\end{itemize}
		L'intensità di ogni fenomeno climatico viene mirata su una scala che va da \textbf{1} (\textit{critico}) a \textbf{5} (\textit{ottimale}).
		\\
		Possono inoltre essere presenti delle note testuali (di max 256 caratteri) per descrivere con più precisione i dati geografici.
		\\
		L'applicazione permette:
		\begin{itemize}
		\item ai \textbf{comuni cittadini}, di visualizzare i parametri climatici di proprio interesse in forma aggregata relativi a ciascuna area
		\item a \textbf{operatori autorizzati}, di gestire aree di interesse, inserendo i vari parametri climatici.
		\end{itemize}
		In particolare questi ultimi hanno la possibilità di registrarsi all'applicazione e successivamente creare centri di monitoraggio ed aggiungervi aree di interesse.

	\newpage

	\section{Avviare l'applicazione}

		\subsection{Requisiti minimi\index{Requisiti minimi}}

		\subsection{Avviare l'applicazione}

		\subsection{Troubleshooting}

	\section{Schermata principale}

	\section{Menù principale}
	\subsection{Voci disponibili}
	\subsection{Funzionamento}
			\subsubsection{Esci}

	\section{Cerca le Misurazioni}
	\section{Login}
	\section{Registrazione}

	\chapter{Limiti della soluzione sviluppata}
	\chapter{Risoluzione problemi comuni}

	\nocite{IuriTex}
	\bibliographystyle{alpha}
	\bibliography{bib/biblio}
	\printindex

\end{document}



