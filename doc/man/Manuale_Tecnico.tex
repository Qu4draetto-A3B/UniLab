\documentclass[a4paper, 12pt]{scrreprt}

\usepackage{sty/preamble}
\usepackage{sty/preamble-math}
\usepackage{sty/preamble-code}

\title {Manuale Tecnico}
\subtitle{CLIMATE MONITORING}
\author{
	\\Iuri Antico \textsl{matricola}:
	\texttt{753144}
	\and \\
	Beatrice Balzarini \textsl{matricola}:
	\texttt{752257}
	\and \\
	Michael Bernasconi \textsl{matricola}:
	\texttt{752259}
	\and \\
	Gabriele Borgia \textsl{matricola}:
	\texttt{753262}\\\\
	}


\date{\today}

\begin{document}

	\maketitle
	\tableofcontents

	\chapter{Introduzione}
	\textsl{Climate Monitoring} \`e un progetto sviluppato nell’ambito del progetto di Laboratorio A (A.A. 2022/23) per il 
	corso di laurea in Informatica dell’\textsl{Universit\`a degli Studi dell’Insubria} di Varese.
	
	Il progetto \`e sviluppato in \textsl{Java 17} sui sistemi operativi \textsl{Arch Linux} e \textsl{Windows 10}.
	L'applicazione \`e stata testata sugli stessi sistemi.
		\section{Librerie esterne utilizzate}
		Per lo sviluppo di questo progetto sono state utilizzate alcune librerie di terze parti.
			\subsection{\textsl{Apache Commons CSV}}
			\`E una libreria di Apache che fornisce i metodi per la gestione dei file \textsl{CSV}. \\In particolare \`e stata utilizzata per la lettura e scrittura dei dati da memorizzare.
			\subsection{\textsl{Apache Commons CLI}}
			\`E una libreria di Apache che fornisce i metodi per la gestione della linea di comando all'interno del terminale.

	\chapter{Struttura del sistema di classi}
	Il codice dell'applicazione \`e suddiviso in diversi package che si occupano di gestire vari aspetti dell'applicazione. In particolare:
	\begin{itemize}
		\item \textbf{\textsl{cli}}\\Contiene classi che gestiscono l'interfaccia da terminale.
		\item \textbf{\textsl{gestori}}\\Contiene classi che si occupano delle operazioni di lettura e scrittura su file.
		\item \textbf{\textsl{magazzeno}}\\Contiene classi che costituiscono le principali istanze di cui si occupa l'applicazione.
		\item \textbf{\textsl{utils}}\\Contiene classi e interfacce che gestiscono una serie di operazioni e funzioni utilizzate in molte classi dell'applicazione.
	\end{itemize}
	
	Verranno ora presentate le classi e le interfacce dell'applicazione nel dettaglio. \\Per informazioni più dettagliate consultare la \textsl{JavaDoc} o i codici sorgente, consultabili tramite qualsiasi \textsl{IDE}.
	\pagebreak
		\section{Package cli}
			\subsection{Class App}
			Classe che consiste nel punto di partenza dell'applicazione.
			\begin{itemize}
				\item \textbf{void start(CommandLine line)}
				\\Metodo che preleva gli parametri forniti dall'applicazione e avvia il comando corretto all'interno di essa.
				
				La complessit\`a del metodo \`e $\bigO\textbf{(1)}$.
			\end{itemize}
			
			\subsection{Class ComandoCentri}
			Classe che inizializza il comando finalizzato alla creazione di nuovi centri di monitoraggio.
			\begin{itemize}
				\item \textbf{void start(Terminal term)}
				\\Metodo che preleva i dati relativi al nuovo centro di monitoraggio da un file "\textsl{centro.ini}" per inizializzare una nuova istanza.
				
				La complessit\`a del metodo \`e $\bigT\textbf{(n)}$.
				
			\end{itemize}
			
			\subsection{Class ComandoMisurazioni}
			Classe che inizializza il comando finalizzato alla creazione di nuove misurazioni.
			\begin{itemize}
				\item \textbf{void start(Terminal term)}
				\\Metodo che preleva i dati relativi alla nuova misurazione da un file "\textsl{misurazione.ini}" per inizializzare una nuova istanza.
				
				La complessit\`a del metodo \`e $\bigO\textbf{(n)}$.
				
			\end{itemize}
			
			\subsection{Class MostraAree}
			Classe che permette di visualizzare nel terminale aree geografiche in base alle coordinate o al geoID.
			\begin{itemize}
				\item \textbf{void start(Terminal term)}
				\\Metodo che recupera dal database le aree geografiche corrispondenti alle coordinate o ai geoID forniti nella linea di comando e le stampa nel terminale.
				
				La complessit\`a del metodo \`e $\bigT\textbf{(n + m + k)}$.
				
			\end{itemize}
			
			\subsection{Class MostraCentri}
			Classe che permette di visualizzare nel terminale centri di monitoraggio in base al nome.
			\begin{itemize}
				\item \textbf{void start(Terminal term)}
				\\Metodo che recupera dal database i centri di monitoraggio corrispondenti al nome fornito nella linea di comando e li stampa nel terminale.
				
				La complessit\`a del metodo \`e $\bigO\textbf{(n)}$.
				
			\end{itemize}
			
			\subsection{Class MostraMisurazioni}
			Classe che permette di visualizzare nel terminale misurazioni in base a determinati criteri forniti.
			\begin{itemize}
				\item \textbf{void start(Terminal term)}
				\\Metodo che recupera dal database le misurazioni corrispondenti a criteri forniti nella linea di comando e le stampa nel terminale.
				
				La complessit\`a del metodo \`e $\bigT\textbf{(n + m)}$.
				
			\end{itemize}
			\subsection{Class MostraUtente}
			Classe che permette di visualizzare nel terminale profili utente in base ai loro userID e password.
			\begin{itemize}
				\item \textbf{void start(Terminal term)}
				\\Metodo che recupera dal database i profili utente corrispondenti a userID e password  forniti nella linea di comando e li stampa nel terminale.
				
				La complessit\`a del metodo \`e $\bigO\textbf{(n\ap{2})}$.
				
			\end{itemize}
			\subsection{Class Registrazione}
			Classe che permette a un utente di eseguire la registrazione all'applicazione.
			\begin{itemize}
				\item \textbf{void start(Terminal term)}
				\\Metodo che permette a un operatore di effettuare la registrazione tramite terminale.
				
				La complessit\`a del metodo \`e $\bigO\textbf{(1)}$.
				
				\textsl{Per ulteriori informazioni sulla mancata implementazione della registrazione consultare il primo punto del paragrafo \textbf{2.2.7}.}
				
			\end{itemize}
		\pagebreak
		\section{Package gestori}
		Package che contiene una serie di classi finalizzate alle operazioni di lettura e scrittura su file riguardanti dati utili al monitoraggio di parametri climatici sul territorio italiano.\\
		Ad ogni record memorizzato su file viene associato un indice (ID) univoco.
			\subsection{Class DataBase}
			Classe che si occupa di creare (per ogni classe che estende Gestore) un oggetto in grado di richiamare le funzioni associate.
			\subsection{Class Gestore}
			Classe che contiene metodi relativi la gestione dei file contenenti i dati d'interesse. In particolare:
			\begin{itemize}
				\item \textbf{void close()}
				\\Metodo che si occupa di chiudere un file.
				
				La complessit\`a del metodo \`e $\bigO\textbf{(1)}$.
				
				\item \textbf{void reload()}
				\\Metodo che si occupa di ricaricare (chiudere e riaprire) un file.
				
				La complessit\`a del metodo \`e $\bigO\textbf{(1)}$.

				\item \textbf {DataTable buildObject (CSVRecord r)}
				\\Metodo astratto che riceve in input come parametro un record e crea l'oggetto associato.
				
				La complessit\`a del metodo \`e $\bigO\textbf{(1)}$.
\pagebreak
				\item \textbf{Result\textless String\textgreater~getProperty(String key)}
				\\Metodo che si occupa di ottenere una propriet\`a nel file\textsl{ CSV.DAT} associato ad una tabella \textsl{CSV}.
				\\Riceve in input come parametro una stringa, che \`e la chiave per ottenere la propriet\`a del file.
				\\Nel caso in cui l'operazione di ricerca venga eseguita correttamente il metodo restituisce un \verb!Result! valido con la stringa della propriet\`a richiesta.
				\\Nel caso in cui l'operazione non venga eseguita correttamente, il metodo restituisce un \verb!Result! che ne indica l'errore.
				
				La complessit\`a del metodo \`e $\bigT\textbf{(n)}$.

				\item \textbf{Result\textless Object\textgreater~setProperty(String key, String val)}
				\\Metodo che si occupa di impostare una propriet\`a nel file \textsl{CSV.DAT} associato ad una tabella \textsl{CSV}.
				\\Il metodo riceve in input due parametri: una stringa che indica la propriet\`a da impostare e la chiave del file.
				\\Nel caso in cui l'operazione non venga eseguita correttamente, il metodo restituisce un \verb!Result! che ne indica l'errore.
				
				La complessit\`a del metodo \`e $\bigT\textbf{(n)}$.
				
			\end{itemize}

			\subsection{Class GestoreArea}
			Classe che estende \verb!Gestore! e ne eredita tutti i metodi.
			\\ Implementa l'interfaccia \textsl{CercaAree} e contiene i seguenti metodi:
			\begin{itemize}
				\item \textbf{Result\textless AreaGeografica\textgreater~getArea(long geoID)}
				\\Metodo che ricerca una determinata area geografica in base al suo ID.
				\\Riceve in input come parametro l'ID dell'area geografica.
				\\Nel caso in cui l'area esista il metodo restituisce un \verb!Result! contenente un l'area geografica, altrimenti una stringa di errore.
				
				La complessit\`a del metodo \`e $\bigO\textbf{(n)}$.
			\end{itemize}

			\subsection{Class GestoreCentro}
			Classe che Gestore, ne eredita tutti i metodi e contiene i seguenti:
			\begin{itemize}
				\item \textbf{Result\textless CentroMonitoraggio\textgreater~getCentro(String nome)}
				\\Metodo che ricerca e restituisce un centro di monitoraggio il cui nome corrisponde alla stringa fornita come parametro .
				
				La complessit\`a del metodo \`e $\bigO\textbf{(n)}$.
				
				\item \textbf {boolean addCentro(CentroMonitoraggio cm)}
				\\Metodo che crea un nuovo record relativo a un determinato centro di monitoraggio fornito come parametro e lo memorizza nel file \textsl{CentriMonitoraggio.CSV}
				\\Il metodo restituisce una variabile booleana che indica se l'operazione \`e stata eseguita correttamente.
				
				La complessit\`a del metodo \`e $\bigO\textbf{(1)}$.
			\end{itemize}

			\subsection{Class GestoreDato}
			Classe che estende Gestore, eredita tutti i metodi e contiene i seguenti:
			\begin{itemize}
				\item \textbf{Result \textless DatoGeografico\textgreater~getDato(long rid)}
				\\Metodo che ricerca un determinato dato geografico in base al suo ID, fornito come parametro.
				\\Restituisce un \verb!Result! contenente il dato geografico se l'operazione viene eseguita correttamente.
				\\In caso contrario il metodo restituisce un \verb!Result! che ne indica l'errore.
				
				La complessit\`a del metodo \`e $\bigO\textbf{(n)}$.
			\pagebreak	
				\item \textbf {Result \textless Object\textgreater~addDato(DatoGeografico dato)}
				\\Metodo che crea un nuovo record relativo a un determinato dato geografico (fornito come parametro) e lo memorizza nel file \textsl{ParametriClimatici.CSV}.
				\\Il metodo restituisce un \verb!Result! contenente il dato geografico se l'operazione viene eseguita correttamente.
				\\In caso contrario il metodo restituisce un \verb!Result! che ne indica l'errore.
				
				La complessit\`a del metodo \`e $\bigO\textbf{(1)}$.
				
			\end{itemize}

			\subsection{Class GestoreMisurazioni}
			Classe che estende Gestore, eredita tutti i metodi e contiene i seguenti:
			\begin{itemize}
				\item \textbf{Result\textless Object\textgreater~addMisurazione(Misurazione mis)}
				\\Metodo che inserisce una nuova misurazione basandosi sul parametro fornito in input.
				\\Il metodo restituisce un \verb!Result! contenente la misurazione se l'operazione viene eseguita correttamente.
				\\In caso contrario il metodo restituisce un \verb!Result! che ne indica l'errore.
				
				La complessit\`a del metodo \`e $\bigO\textbf{(1)}$.

				\item \textbf {Result \textless Filtratore\textgreater~getMisurazioni()}
				\\Metodo che raccoglie e memorizza i record relativi alle misurazioni presenti nel file \textsl{ParametriClimatici.CSV)}.
				
				La complessit\`a del metodo \`e $\bigT\textbf{(n)}$.
				
			\end{itemize}
\pagebreak
			\subsection{Class GestoreOperatore}
			Classe che estende Gestore, eredita tutti i metodi e contiene i seguenti:
			\begin{itemize}
				\item \textbf{Result\textless Operatore\textgreater~registrazione(Operatore op, String pwd)}
				\\Metodo non implementato, in quanto l'operatore richiede un centro di monitoraggio che il GestoreCentro fallisce nel restituire, impedendo la corretta registrazione dell'utente.
				
				La complessit\`a del metodo \`e $\bigO\textbf{(1)}$.
				
				\item \textbf{Result \textless Operatore\textgreater~login(String uid, String pwd)}
				\\Metodo che permette ad un operatore di effettuare il login.
				\\I parametri forniti in input sono l'ID utente e la password. In caso questi ultimi corrispondano ad un operatore esistente, il metodo permette all'operatore di effettuare l'accesso.
				\\In caso contrario restituisce una stringa relativa a un codice di errore.
				
				La complessit\`a del metodo \`e $\bigO\textbf{(n)}$.
			\end{itemize}

\pagebreak
		\section{Package magazzeno}
		All'interno del package magazzeno si trovano una serie di classi finalizzate a memorizzare informazioni oggetto di operazioni di lettura o scrittura su file \textsl{CSV}.

			\subsection{Class AreaGeografica}
			Classe che implementa l'interfaccia \textsl{DataTable}, e contiene i seguenti metodi:
				\begin{itemize}
				\item \textbf{metodi getter}
				\\getgeoID(), getLatitudine(), getLongitudine(), getStato(), getDenominazione().
				
				La complessit\`a dei metodi \`e $\bigO\textbf{(1)}$.
				
				\item \textbf{String toString()}
				\\Metodo che permette di stampare l'oggetto \verb!AreaGeografica!, mostrando i relativi denominazione, stato, latitudine e longitudine.
				
				La complessit\`a del metodo \`e $\bigO\textbf{(1)}$.
				
				\item \textbf{boolean equals(Object obj)}
				\\Metodo che permette di confrontare un oggetto qualsiasi con un oggetto di tipo \verb!AreaGeografica!.
				\\Riceve in input come parametro un \verb!Object!.
				\\Restituisce \verb!true! se l'oggetto di tipo \verb!Object! \`e un istanza di \verb!AreaGeografica!.
				\\In caso contrario il metodo restituisce \verb!false!.
				
				La complessit\`a del metodo \`e $\bigO\textbf{(1)}$.
				
			\end{itemize}
\pagebreak
			\subsection{Class CentroMonitoraggio}
			Classe che implementa le due interfacce \textsl{Convertable} e \textsl{Datatable} e contiene i seguenti metodi:
			\begin{itemize}
				\item \textbf{metodi getter}
				\\getNome(), getIndirizzo(), getAree().
				
				La complessit\`a dei metodi \`e $\bigO\textbf{(1)}$.
				
				\item \textbf{String toString()}
				\\Metodo che permette di stampare l'oggetto \verb!CentroMonitoraggio!, mostrando i relativi nome, indirizzo e aree geografiche associate.
				
				La complessit\`a del metodo \`e $\bigO\textbf{(1)}$.
				
				\item \textbf{boolean equals(Object obj)}
				\\Metodo che permette di confrontare un oggetto qualsiasi con un oggetto di tipo \verb!CentroMonitoraggio!.
				\\Riceve in input come parametro un \verb!Object!.
				\\Restituisce \verb!true! se l'oggetto di tipo \verb!Object! \`e un istanza di \verb!CentroMonitoraggio!.
				\\In caso contrario il metodo restituisce \verb!false!.
				
				La complessit\`a del metodo \`e $\bigO\textbf{(1)}$.
				
				\item \textbf{String toCSV()}
				\\Metodo che permette di creare una stringa che rappresenta l'oggetto \verb!CentroMonitoraggio! nel formato CSV adoperato all'interno del programma.
				
				La complessit\`a del metodo \`e $\bigO\textbf{(1)}$.
				
			\end{itemize}
\pagebreak
			\subsection{Class DatoGeografico}
			Classe che implementa l'interfaccia \textsl{DataTable} e contiene i seguenti metodi:
			\begin{itemize}
				\item \textbf{void setDato(TipoDatoGeografico tipo, byte dato)}
				\\Metodo che imposta un dato geografico.
				\\Riceve come parametri il tipo di dato da impostare e il suo valore.
				
				La complessit\`a del metodo \`e $\bigO\textbf{(1)}$.
				
				\item \textbf{byte getDato(TipoDatoGeografico tipo)}
				\\Metodo che riceve come parametro in input un'istanza di \verb!TipoDatoGeografico! e restituisce dato geografico associato.
				
				La complessit\`a del metodo \`e $\bigO\textbf{(1)}$.
				
				\item \textbf{String getNota(TipoDatoGeografico key)}
				\\Metodo che preleva la nota del dato geografico che esegue il metodo e restituisce la nota associata alla chiave fornita come parametro.
				
				La complessit\`a del metodo \`e $\bigO\textbf{(1)}$.
				
				\item \textbf{boolean setNota(TipoDatoGeografico key, String nota)}
				\\Metodo che permette di impostare le note relative al dato geografico.
				\\Riceve come parametri una chiave che stabilisce come inserire la nota all'interno del dato geografico e la nota da inserire.
				\\La funzione restituisce \verb!true! se l'operazione viene eseguita correttamente, \verb!false! altrimenti.
				
				La complessit\`a del metodo \`e $\bigO\textbf{(1)}$.
				
				\item \textbf{boolean equals(Object obj)}
				\\Metodo che permette di confrontare un oggetto qualsiasi con un oggetto di tipo \verb!DatoGeografico!.
				\\Riceve in input come parametro un \verb!Object!.
				\\Restituisce \verb!true! se l'oggetto di tipo \verb!Object! \`e un istanza di \verb!DatoGeografico!.
				\\In caso contrario il metodo restituisce \verb!false!.
				
				La complessit\`a del metodo \`e $\bigO\textbf{(1)}$.
				\pagebreak
				\item \textbf{String toString()}
				\\Metodo che permette di stampare l'oggetto \verb!DatoGeografico!, mostrando i relativi nome, tipo e note.
				
				La complessit\`a del metodo \`e $\bigO\textbf{(1)}$.
				
				\item \textbf{boolean noteEquals(DatoGeografico dato)}
				\\Metodo che riceve in input un oggetto di tipo \verb!DatoGeografico! da cui estrae la relativa nota e la confronta con quella che esegue il metodo.
				\\Restituisce \verb!true! se le due note sono uguali, \verb!false! altrimenti.
				
				La complessit\`a del metodo \`e $\bigO\textbf{(1)}$.
				
				\item \textbf{boolean datoEquals(DatoGeografico dato)}
				\\Metodo che riceve in input come parametro un dato geografico e restituisce \verb!true! se l'uguaglianza dei valori dati \`e verificata, \verb!false! altrimenti.
				
				La complessit\`a del metodo \`e $\bigO\textbf{(1)}$.
				
			\end{itemize}

			\subsection{Class Filtratore}
			Classe che implementa le interfacce: \textsl{Iterable}, \textsl{CercaAree}, \textsl{MediaAree} e contiene i seguenti metodi:
			\begin{itemize}
				\item \textbf{Filtratore filtra(DataTable... dts)}
				\\Metodo che permette di filtrare i record.
				\\Riceve come parametro una serie elementi di tipo \verb!Datatable! e restituisce i record filtrati.
				
				La complessit\`a del metodo \`e $\bigO\textbf{(n$\cdot$m)}$.
				
				\item \textbf{Filtratore filtraOperatore(Operatore... ops)}
				\\Metodo che permette di filtrare gli operatori.
				\\Riceve come parametro una serie di elementi di tipo Operatore e restituisce gli operatori filtrati.
				
				La complessit\`a del metodo \`e $\bigO\textbf{(n$\cdot$m)}$.
				
				\pagebreak
				\item \textbf{Filtratore filtraCentro(CentroMonitoraggio... cms)}
				\\Metodo che permette di filtrare i centri di monitoraggio.
				\\Riceve come parametro una serie di centri di monitoraggio e restituisce i centri di monitoraggio filtrati.
				
				La complessit\`a del metodo \`e $\bigO\textbf{(n$\cdot$m)}$.
				
				\item \textbf{Filtratore filtraAree(AreaGeografica... ags)}
				\\Metodo che permette di filtrare le aree geografiche.
				\\Riceve come parametro una serie di aree geografiche e restituisce le aree geografiche filtrate.
				
				La complessit\`a del metodo \`e $\bigO\textbf{(n$\cdot$m)}$.
				
				\item \textbf{Filtratore filtraNote(String... note)}
				\\Metodo che permette di filtrare le note.
				\\Riceve come parametro una serie di elementi di tipo \verb!String! e restituisce le note filtrate.
				
				La complessit\`a del metodo \`e $\bigO\textbf{(n$\cdot$m)}$.
				
				\item \textbf {Filtratore filtraDato(DatoGeografico... dati)}
				\\Metodo che permette di filtrare i dati geografici.
				\\Riceve come parametro una serie di elementi dati geografici e restituisce i dati geografici filtrati.
				
				La complessit\`a del metodo \`e $\bigO\textbf{(n$\cdot$m)}$.
				
				\item \textbf {String toString()}
				\\Metodo che stampa l'oggetto \verb!Filtratore!.
				
				La complessit\`a del metodo \`e $\bigT$\textbf{(n)}.
				
				\item \textbf{DatoGeografico visualizzaAreaGeografica(AreaGeografica area)}
				\\Metodo che riceve in input come parametro un'area geografica.
				\\Conta quante volte appare un valore dell'Area Geografica e restituisce un dato geografico il valore associato.
				
				La complessit\`a del metodo \`e $\bigT$\textbf{(n)}.
				
				\item \textbf{Iterator<Misurazione> iterator()}
				\\Metodo che restituisce un \verb!Iterator! contenente le misurazioni.
				
				La complessit\`a del metodo \`e $\bigO\textbf{(1)}$.
				
			\end{itemize}
			\subsection{Class Indirizzo}
			Classe che definisce l'indirizzo dei centri di monitoraggio.
			\\Contiene i seguenti metodi:\\
			\begin{itemize}
				\item \textbf{metodi getter}
				\\ getNomeVia(), getCivico(), getCap(), getComune(), getProvincia().
				
				La complessit\`a dei metodi \`e $\bigO\textbf{(1)}$.
				
				\item \textbf{String toString()}
				\\Metodo che permette di stampare l'oggetto \verb!Indirizzo!, mostrando i relativi nome della via, numero civico, cap, comune e provincia..
				
				La complessit\`a del metodo \`e $\bigO\textbf{(1)}$.
				
				\item \textbf{String toCsv()}
				\\Metodo che permette di creare una stringa che descrive l'indirizzo nel formato \textsl{CSV} adoperato all'interno del programma.
				
				La complessit\`a del metodo \`e $\bigO\textbf{(1)}$.
				
			\end{itemize}
\pagebreak
			\subsection{Class ListaAree}
			Classe che implementa le interfacce \textsl{Iterable}, \textsl{CercaAree} e \textsl{Convertable} e contiene i seguenti metodi:
			\begin{itemize}
				\item \textbf{boolean isEmpty()}
				\\Metodo che restituisce \textbf{true} se la lista di aree geografiche che esegue \`e vuota, \verb!false! altrimenti.
				
				La complessit\`a del metodo \`e $\bigO\textbf{(1)}$.
				
				\item \textbf{AreaGeografica get(int k)}
				\\Metodo che riceve in input un intero k che indica la posizione di un'area geografica all'interno della lista di aree geografiche e restituisce l'area geografica che \`e in posizione k.
				
				La complessit\`a del metodo \`e $\bigO\textbf{(n)}$.
				
				\item \textbf{void add(AreaGeografica e, int k)}
				\\Metodo che aggiunge l'area geografica in posizione k.
				\\Riceve in input un'area geografica e un intero k.
				
				La complessit\`a del metodo \`e $\bigO\textbf{(n)}$.
				
				\item \textbf{AreaGeografica getFirst()}
				\\Metodo che restituisce il primo elemento presente nella lista di aree geografiche che esegue il metodo.
				
				La complessit\`a del metodo \`e $\bigO\textbf{(1)}$.
				
				\item \textbf{AreaGeografica getLast()}
				\\Metodo che restituisce l'ultimo elemento presente nella lista di aree geografiche che esegue.
				
				La complessit\`a del metodo \`e $\bigO\textbf{(1)}$.
				\pagebreak
				\item \textbf{int size()}
				\\Metodo che restituisce un intero che indica la dimensione della lista di aree geografiche che esegue il metodo.
				
				La complessit\`a del metodo \`e $\bigO\textbf{(n)}$.
				
				\item \textbf{void addFirst(AreaGeografica e)}
				\\Metodo che aggiunge in prima posizione della lista di aree geografiche che esegue il metodo un'area geografica, fornita in input come parametro.
				
				La complessit\`a del metodo \`e $\bigO\textbf{(1)}$.
				
				\item \textbf{Iterator<AreaGeografica> iterator()}
				\\Metodo che restituisce un \verb!Iterator! di aree geografiche.
				
				La complessit\`a del metodo \`e $\bigO\textbf{(1)}$.
				
				\item \textbf{ListaAree cercaAreaGeografica(String denominazione, String stato)}
				\\Metodo che ricerca le aree geografiche mediante i parametri denominazione e stato e restituisce il risultato della ricerca in una lista di aree geografiche.
				
				La complessit\`a del metodo \`e $\bigO\textbf{(n)}$.
				
				\item \textbf{Result<AreaGeografica> cercaAreeGeografiche(double latitudine, double longitudine)}
				\\Metodo che ricerca le aree geografiche mediante i parametri forniti in input, latitudine e longitudine e restituisce un \verb!Result! di aree geografiche.
				
				La complessit\`a del metodo \`e $\bigO\textbf{(n)}$.
				
				\item \textbf{String toString()}
				\\Metodo che restituisce una stringa che rappresenta una lista di aree geografiche.
				
				La complessit\`a del metodo \`e $\bigO\textbf{(n)}$.
				\pagebreak
				\item \textbf{String toCsv()}
				\\Metodo che permette di creare una stringa per descrivere l'area geografica che esegue nel formato \textsl{CSV} adoperato all'interno del programma.
				
				La complessit\`a del metodo \`e $\bigO\textbf{(n)}$.
				
				
				\item \textbf{Result<AreaGeografica> getArea(long geoId)}
				\\Metodo recupera l'area tramite il suo geoID.
				
				La complessit\`a del metodo \`e $\bigO\textbf{(n)}$.
				
			\end{itemize}

			\subsection{Class Misurazioni}
			Classe che implementa le interfacce \textsl{Convertable} e \textsl{DataTable} e contiene i seguenti metodi:\\
			\begin{itemize}
				\item \textbf{metodi getter}
				\\getRid(), getDato(), getTime(), getTimeString(), getOperatore(), getCentro(), getArea().
				
				La complessit\`a dei metodi \`e $\bigO\textbf{(1)}$.
				
				\item \textbf{String toString()}
				\\Metodo che permette di creare una stringa dell'oggetto \verb!Misurazione!, mostrando i relativi data e ora, area geografica, operatore associato, centro di monitoraggio associato e dato geografico.
				
				La complessit\`a dei metodi \`e $\bigO\textbf{(1)}$.
				
			\end{itemize}
\pagebreak
			\subsection{Class Operatore}
			Classe che definisce i seguenti metodi:\\
			\begin{itemize}
				\item \textbf{metodi getter}
				\\getCf(), getCentro(), getCognome(), getNome(), getEmail(), getUid().
				
				La complessit\`a dei metodi \`e $\bigO\textbf{(1)}$.
				
				\item \textbf{Result<Object> inserisciParametri(AreaGeografica area, DatoGeografico dato, LocalDateTime tempo)}
				\\Metodo che consente di inserire i dati climatici di una determinata area nel database i cui dati vengono forniti in input come parametri e restituisce un \verb!Result! di Object.
				
				La complessit\`a del metodo \`e $\bigO\textbf{(1)}$.
				
				\item \textbf{String toString()} e \textbf{String toStringPretty()}
				\\Metodi che permettono di stampare l'oggetto \verb!Operatore!, mostrando i relativi codice fiscale, userID, nome, cognome, e-mail, e centro di monitoraggio associato.
				
				La complessit\`a del metodo \`e $\bigO\textbf{(1)}$.
				
				\item \textbf{String toCsv()}
				\\Metodo che crea una stringa relativa all'operatore che esegue nel formato \textsl{CSV} adoperato all'interno del programma.
				
				La complessit\`a del metodo \`e $\bigO\textbf{(1)}$.
				
				\item \textbf{boolean equals(Object obj)}
				\\Metodo che permette di confrontare un oggetto qualsiasi con un oggetto di tipo \verb!Operatore!.
				\\Riceve in input come parametro un \verb!Object!.
				\\Restituisce \verb!true! se l'oggetto di tipo \verb!Object! \`e un istanza di \verb!Operatore!.
				\\In caso contrario il metodo restituisce \verb!false!.
				
				La complessit\`a del metodo \`e $\bigO\textbf{(1)}$.
				
			\end{itemize}
\pagebreak
		\section{Package utils}
			\subsection{Package listacustom}
			Package che contiene classi utilizzate come supporto alla classe ListaAree.
				\subsubsection{Class CollezioniIterator}
				Classe che permette a ListaAree di svolgere l'istruzione "for-each loop". Implementa l'interfaccia \textsl{Iterator}.
				\begin{itemize}
					\item\textbf{E next()}
					\\Metodo che restituisce l'elemento corrente e scorre a quello successivo.
					
					La complessit\`a del metodo \`e $\bigO\textbf{(1)}$.
					
					\item\textbf{boolean hasNext()}
					\\Metodo che restituisce \verb!true! se il nodo che esegue il metodo ha un successore, \verb!false! altrimenti.
					
					La complessit\`a del metodo \`e $\bigO\textbf{(1)}$.
					
				\end{itemize}

				\subsubsection{Class Nodo}
				Classe che rappresenta i nodi della lista i cui elementi sono gestiti dai seguenti metodi:\\
				\begin{itemize}
					\item\textbf{setter}
					\\setDato(), setNext().
					
					La complessit\`a dei metodi \`e $\bigO\textbf{(1)}$.
					\pagebreak
					\item\textbf{metodi getter}
					\\getDato(), getNext().
					
					La complessit\`a dei metodi \`e $\bigO\textbf{(1)}$.
					
				\end{itemize}

			\subsection{Package Result}
				All'interno del package Result sono presenti una serie di classi finalizzate a gestire i risultati di alcuni metodi dell'applicazione.
				
				\subsubsection{Class Panic}
				Classe che estende Error, finalizzata a gestire degli errori lanciati dalla classe \verb!Result! che non \`e possibile catturare.

				\subsubsection{Class Result}
				Classe che si occupa della gestione dei risultati in alcuni metodi che potrebbero lanciare errori nell'applicazione.\\
				\begin{itemize}
					\item \textbf{metodi getter}
					\\getError(), getMessage(), getFullMessage().
					
					La complessit\`a dei metodi \`e $\bigO\textbf{(1)}$.
					
					\item\textbf{boolean isValid()}
					\\Metodo che restituisce \verb!true! se il \verb!Result! \`e valido, \verb!false! altrimenti.
					
					La complessit\`a del metodo \`e $\bigO\textbf{(1)}$.
					
					\item\textbf{boolen isError()}
					\\Metodo che restituisce \verb!true! se il \verb!Result! lancia un errore, \verb!false! altrimenti.
					
					La complessit\`a del metodo \`e $\bigO\textbf{(1)}$.
	
					\item\textbf{void ifValid(BiConsumer\textless T, Integer \textgreater~fn)}
					\\Metodo che esegue la funzione data come parametro se il \verb!Result! \`e valido.
					
					La complessit\`a del metodo \`e $\bigO\textbf{(1)}$.
	
					\item\textbf{void ifError(BiConsumer\textless T, Integer\textgreater~fn)}
					\\Metodo che esegue la funzione data come parametro se il \verb!Result! genera errore.
					
					La complessit\`a del metodo \`e $\bigO\textbf{(1)}$.
	
					\item \textbf{T get()}
					\\Metodo che restituisce il contenuto di \verb!Result!.
					
					La complessit\`a del metodo \`e $\bigO\textbf{(1)}$.
	
					\item \textbf{T getOr(T other)}
					\\Metodo che restituisce il contenuto di \verb!Result! se questo non \`e nullo.
					\\ In caso contrario restituisce il parametro other.
					
					La complessit\`a del metodo \`e $\bigO\textbf{(1)}$.
	
					\item \textbf{T getOrElse(Supplier \textless T\textgreater~fn)}
					\\Metodo che restituisce il contenuto di \verb!Result! se questo non \`e nullo.
					\\In caso contrario esegue la funzione fornita come parametro e restituisce il risultato di quest'ultima.
					
					La complessit\`a del metodo \`e $\bigO\textbf{(1)}$.
	
					\item \textbf{T except()}
					\\Metodo che restituisce il contenuto di \verb!Result! senza eseguire nessun controllo.
					
					La complessit\`a del metodo \`e $\bigO\textbf{(1)}$.
	
					\item \textbf{void panic()}
					\\Metodo che lancia un errore non catturabile.
					
					La complessit\`a del metodo \`e $\bigO\textbf{(1)}$.
					
				\end{itemize}

			\subsection{Package terminal}
			Package che contiene classi finalizzate alla gestione del terminale.
				\subsubsection{Class Screen}
				Classe che contiene il seguente metodo:
				\begin{itemize}
					\item \textbf{void show(View v)}
					\\Metodo finalizzato a mostrare View.
					\\Pulisce il terminale prima e dopo l'esecuzione dell'applicazione.
					
					La complessit\`a del metodo \`e $\bigO\textbf{(1)}$.
					
				\end{itemize}

				\subsubsection{Class Terminal}
				Classe Involucro che racchiude System.in() e System.out(), aggiungendo varie funzionalit\`a.
				\begin{itemize}
					\item \textbf{void clear()}
					\\Metodo che pulisce la console con il codice di uscita ANSI.
					
					La complessit\`a del metodo \`e $\bigO\textbf{(1)}$.
					
					\item \textbf{void printf(String str, Object... args)}
					\\Metodo che formatta e stampa una stringa nel terminale.
					\\I parametri formali servono a stampare all'utente una stringa (str) interpolata (args).
					
					La complessit\`a del metodo \`e $\bigO\textbf{(1)}$.
					
					\item \textbf{void printfln(String str, Object... args)}
					\\Metodo che formatta e stampa una stringa nel terminale.
					\\I parametri formali sono finalizzati a stampare all'utente una stringa (str) interpolata (args).
					
					La complessit\`a del metodo \`e$\bigO\textbf{(1)}$.
					
					\item \textbf{String readLine()}
					\\Metodo che permette di leggere una linea dalla console utente e restituisce la stringa inserita dall'utente.
					
					La complessit\`a del metodo \`e $\bigO\textbf{(1)}$.
					
					\item \textbf{String readLine(String str, Object... args)}
					\\Metodo che stampa una stringa e aspetta una risposta dell'utente.
					\\I parametri formali sono finalizzati a stampare all'utente una stringa (str) interpolata (args).
					
					La complessit\`a del metodo \`e $\bigO\textbf{(1)}$.
					
					\item \textbf{String readLineOrDefault(String def, String str, Object... args)}
					\\Metodo che stampa una stringa all'utente e aspetta una risposta, se quest'ultima non viene data (stringa vuota), il metodo restituisce (def).
					\\I parametri formali sono finalizzati a stampare all'utente una stringa (str) interpolata (args).
					
					La complessit\`a del metodo \`e $\bigO\textbf{(1)}$.
					
					\item \textbf{String readWhile(Predicate<String> fn, String str, Object... args)}
					\\Stampa una stringa e aspetta una risposta dall'utente, controllando la stringa fornita come parametro.
					\\I parametri formali sono finalizzati a stampare all'utente una stringa (str) interpolata (args). In particolare fn \`e una funzione che restituisce un booleano.
					
					La complessit\`a del metodo \`e $\bigO\textbf{(n)}$.
	
					\item \textbf{boolean promptUser(boolean yes, String str, Object... args)}
					\\Metodo che pone all'utente una domanda con risposta di tipo si/no.
					\\I parametri formali sono finalizzati a stampare all'utente una stringa (str) interpolata (args). In particolare rappresenta la scelta di default.
					
					La complessit\`a del metodo \`e $\bigO\textbf{(1)}$.
					
				\end{itemize}

				\subsubsection{Interface View}
				Interfaccia che contiene il seguente metodo:
				\begin{itemize}
					\item \textbf{abstract void start(Terminal term)}
					\\Metodo astratto finalizzato a avviare una schermata dell'applicazione attraverso il terminale.
					
					La complessit\`a del metodo \`e $\bigO\textbf{(1)}$.
				\end{itemize}

			\subsection{Interface CercaAree}
			L'implementazione di questa interfaccia consente la ricerca di aree geografiche.
			\begin{itemize}
				\item \textbf{ListaAree cercaAreaGeografica (String denominazione, String stato)}
				\\Metodo che ricerca delle aree geografiche mediante denominazione e stato di appartenenza.
				\\Restituisce un \verb!Result! contenente le aree geografiche corrispondenti al risultato della ricerca.
				
				\item\textbf{Result\textless AreaGeografica\textgreater~cercaAreeGeografiche(double latitudine, double longitudine)}
				\\Metodo che ricerca delle aree geografiche mediante coordinate (latitudine e longitudine fornite in input come parametro).
				\\Restituisce un \verb!Result! contenente le aree geografiche corrispondenti al risultato della ricerca.
				
			\end{itemize}

			\subsection{Interface Convertable}
			L'implementazione di questa interfaccia consente a un oggetto di essere convertito nel formato \textsl{CSV}.
			\begin{itemize}
				\item \textbf{String toCsv()}
				\\Metodo che converte l'oggetto nel formato \textsl{CSV} e restituisce una stringa relativa all'oggetto.
				
			\end{itemize}
			\subsection{Interface DataTable}
			L'implementazione di questa interfaccia permette di confrontare due record.
			\begin{itemize}
				\item\textbf{boolean equals(Object obj)}
				\\Metodo che permette di confrontare un oggetto qualsiasi con un oggetto di tipo \verb!DataTable!.
				\\Riceve in input come parametro un \verb!Object!.
				\\Restituisce \verb!true! se l'oggetto di tipo \verb!Object! \`e un istanza di \verb!DataTable!.
				\\In caso contrario il metodo restituisce \verb!false!.
				
				La complessit\`a del metodo \`e $\bigO\textbf{(1)}$.
				
			\end{itemize}
			\subsection{Interface MediaAree}
			L'nterfaccia MediaAree permette di visualizzare le informazioni relative ad un'area geografica.
			\begin{itemize}
				\item\textbf {DatoGeografico visualizzaAreaGeografica (AreaGeografica area)}
				\\Metodo che restituisce un nuovo dato geografico, che rappresenta un prospetto riassuntivo dei parametri climatici associati all'area geografica fornita in input.				
			\end{itemize}
			\pagebreak
			\subsection{Class IniFile}
			Classe che si occupa di leggere un file \textsl{.ini} ed eseguirne il parsing.
			\begin{itemize}
				\item \textbf{void load(String path) throws IOException}
				\\Metodo che carica il file \textsl{.ini}.
				
				La complessit\`a del metodo \`e $\bigT\textbf{(n)}$
				
				\item \textbf{String getString(String section, String key, String defaultvalue)}
				\\Metodo che preleva un valore identificato in base a chiave e a sezione fornite in input come parametri.
				
				La complessit\`a del metodo \`e $\bigO\textbf{(1)}$
				
				\item \textbf{int getInt(String section, String key, int defaultvalue)}
				\\Metodo che preleva il valore identificato dalla chiave e dalla sezione date come parametri e lo prova a convertire ad intero.
				
				La complessit\`a del metodo \`e $\bigO\textbf{(1)}$
				
				\item \textbf{double getDouble(String section, String key, double defaultvalue)}
				\\Metodo che preleva il valore identificato in base a chiave e a sezione fornite in input come parametri e lo converte a \textsl{double}.
				
				La complessit\`a del metodo \`e $\bigO\textbf{(1)}$
				
				
			\end{itemize}
			\subsection{Enum TipoDatoGeografico}
			Enumerativo che rappresenta il tipo di un dato geografico.


		\section{Class Main}
		Entry point dell'applicazione.

	\nocite{IuriTex}
	\bibliographystyle{alpha}
	\bibliography{bib/biblio}
	\printindex

\end{document}



