\documentclass[a4paper, 12pt]{scrreprt}

\usepackage{sty/preamble}
\usepackage{sty/preamble-math}
\usepackage{sty/preamble-code}

\title {Manuale Tecnico}
\subtitle{CLIMATE MONITORING}
\author{
	\\Iuri Antico \textit{matricola}:
	\texttt{753144}
	\and \\
	Beatrice Balzarini \textit{matricola}:
	\texttt{752257}
	\and \\
	Michael Bernasconi \textit{matricola}:
	\texttt{752259}
	\and \\
	Gabriele Borgia \textit{matricola}:
	\texttt{753262}
}

\date{\today}

\begin{document}

	\maketitle

	\tableofcontents
	\listoffigures
	\listoftables

	\chapter{Introduzione}
		\section{Librerie esterne utilizzate}
			\subsection{Apache Commons CSV}
			\`E una libreria di Apache che fornisce i metodi per la gestione dei file (*.CSV). In particolare \`e stata usata nel progetto per la lettura e scrittura dei dati da memorizzare.
			\subsection{Apache Commons CLI}
			\`E una libreria di Apache che fornisce i metodi per la gestione della linea di comando all'interno del terminale.

	\chapter{Struttura generale del sistema di classi}
		\section{cli}
			\subsection{App}
			La classe App
			\begin{itemize}
				\item \textbf{void start(CommandLine line)}
				\\Metodo
				
				La complessit\`a del metodo è
				
			\end{itemize}
			
			\subsection{ComandoAree}
			La classe ComandoAree
			\begin{itemize}
				\item \textbf{void start(Terminal term)}
				\\Metodo
				
				La complessit\`a del metodo è
				
			\end{itemize}
			\subsection{ComandoCentri}
			La classe ComandoCentri
			\begin{itemize}
				\item \textbf{void start(Terminal term)}
				\\Metodo
				
				La complessit\`a del metodo è
				
			\end{itemize}
			\subsection{ComandoMisurazioni}
			La classe ComandoMisurazioni
			\begin{itemize}
				\item \textbf{void start(Terminal term)}
				\\Metodo
				
				La complessit\`a del metodo è 
				
			\end{itemize}
			\subsection{ComandoOperatori}
			La classe ComandoOperatori
			\begin{itemize}
				\item \textbf{void start(Terminal term)}
				\\Metodo
				
				La complessit\`a del metodo è 
				
			\end{itemize}
			\subsection{MostraCentri}
			La classe ComandoMisurazioni
			\begin{itemize}
				\item \textbf{void start(Terminal term)}
				\\Metodo
				
				La complessit\`a del metodo è 
				
			\end{itemize}
			
			\subsection{MostraMisurazioni}
			La classe ComandoMisurazioni
			\begin{itemize}
				\item \textbf{void start(Terminal term)}
				\\Metodo
				
				La complessit\`a del metodo è 
				
			\end{itemize}
			\subsection{Query}
			La classe Query si occupa di 
			\begin{itemize}
				\item \textbf{void start(Terminal term)}
				\\Metodo 
				
				La complessit\`a del metodo è 
				
			\end{itemize}
			\subsection{Registrazione}
			La classe registrazione
			\begin{itemize}
				\item \textbf{void start(Terminal term)}
				\\Metodo
				
				La complessit\`a del metodo è 
				
			\end{itemize}
			
		\section{gestori}
		All'interno del package Gestori si trovano una serie di classi finalizzate alle operazioni di lettura e scrittura su File contenenti dati utili al monitoraggio di parametri climatici sul territorio italiano.
		Ad ogni record memorizzato su file viene associato un indice (ID) univoco.
			\subsection{DataBase}
			La classe DataBase si occupa di creare (per ogni classe che estende Gestore) un oggetto in grado di richiamare le funzioni associate.
			\subsection{Gestore}
			La classe astratta Gestore contiene metodi relativi la gestione dei file contenenti i dati d'interesse. In particolare:
			\begin{itemize}
				\item \textbf{void close()}
				\\Metodo che si occupa di chiudere un file.
				
				La complessit\`a del metodo è \textbf{O(1)}.
				
				\item \textbf{void reload()}
				\\Metodo che si occupa di ricaricare (chiudere e riaprire un file).
				
				La complessit\`a del metodo è \textbf{O(1)}.

				\item \textbf {DataTable buildObject (CSVRecord r)}
				\\Metodo astratto che si occupa di prendere in input come parametro formale un record e creare l'oggetto associato a chi implementa il metodo.
				
				La complessit\`a del metodo è \textbf{O(1)}.

				\item \textbf{Result\textless String\textgreater getProperty(String key)}
				\\Metodo che si occupa di ottenere una propriet\`a nel file (*.CSV.DAT) associato ad una tabella (*.CSV). Il parametro formale di questo metodo \`e una stringa, che \`e la chiave per ottenere la propriet\`a del file.
				\\Nel caso in cui l'operazione di ricerca venga eseguita correttamente il metodo restituisce un Result valido con la stringa della propriet\`a richiesta.
				\\Nel caso in cui l'operazione non venga eseguita in maniera corretta, il metodo restituisce una stringa associata ad un codice di errore.
				
				La complessit\`a del metodo \`e $\bigT$\textbf{(n)}.

				\item  \textbf{Result\textless Object\textgreater setProperty(String s, String k)}
				\\Metodo che si occupa di impostare una propriet\`a nel file (*.CSV.DAT) associato ad una tabella (*.CSV).
				\\Il metodo riceve in input due parametri formali: una stringa che indica la propriet\`a da impostare e la chiave del file.
				\\Nel caso in cui l'operazione non venga eseguita correttamente, il metodo restituisce un Result di Object come errore.
				
				La complessit\`a del metodo è \textbf{O(1)}.
				
			\end{itemize}

			\subsection{GestoreArea}
			La classe GestoreArea estende Gestore e ne eredita tutti i metodi.
			\\ Implementa l'interfaccia CercaAree e contiene i seguenti metodi:
			\begin{itemize}
				\item \textbf{Result\textless AreaGeografica\textgreater getArea(long geoID)}
				\\Metodo che ricerca una determinata area geografica in base al suo ID. In questo metodo c'\`e un solo parametro formale, il numero di geoID.
				\\Nel caso in cui l'area esista il metodo restituisce un Result di AreaGeografica, altrimenti una stringa di errore.
				
				La complessit\`a del metodo \`e \textbf{O(n)}.
			\end{itemize}

			\subsection{GestoreCentro}
			La classe GestoreCentro estende la classe Gestore, ne eredita tutti i metodi e contiene i seguenti:
			\begin{itemize}
				\item \textbf{Result\textless CentroMonitoraggio\textgreater getCentro(String nome)}
				\\Metodo che ricerca e restituisce l'oggetto CentroMonitoraggio il cui nome corrisponde alla stringa fornita come parametro formale.
				
				La complessit\`a del metodo \`e \textbf{O(n)}.
				
				\item \textbf {boolean addCentro(CentroMonitoraggio cm)}
				\\Metodo che crea un nuovo record relativo a un determinato centro di monitoraggio fornito come parametro formale e lo memorizza nel file CentriMonitoraggio (.*CSV).
				\\Il metodo restituisce una variabile booleana che indica se l'operazione \`e stata eseguita correttamente.
				
				La complessit\`a del metodo è \textbf{O(1)}.
			\end{itemize}

			\subsection{GestoreDato}
			La classe GestoreDato estende la classe Gestore, ne eredita tutti i metodi e contiene i seguenti:
			\begin{itemize}
				\item \textbf{Result \textless DatoGeografico\textgreater getDato(long rid)}
				\\Metodo che ricerca un determinato dato geografico in base al suo ID, fornito come parametro formale.
				\\Restituisce un Result di DatoGeografico se l'operazione viene eseguita correttamente.
				\\Altrimenti, nel caso in cui l'operazione non venga eseguita in maniera corretta, il metodo restituisce un Result che ne indica l'errore.
				
				La complessit\`a del metodo \`e \textbf{O(n)}.
				
				\item \textbf {Result \textless Object\textgreater addDato(DatoGeografico dato)}
				\\Metodo che crea un nuovo record relativo a un determinato dato geografico (fornito come parametro formale) e lo memorizza nel file \textit{ParametriClimatici.CSV}.
				\\Il metodo restituisce un Result di Object se l'operazione non \`e stata eseguita correttamente.
				
				La complessit\`a del metodo è \textbf{O(1)}.
				
			\end{itemize}

			\subsection{GestoreMisurazioni}
			La classe GestoreMisurazioni estende la classe Gestore, eredita tutti i metodi e contiene i seguenti:
			\begin{itemize}
				\item \textbf{Result\textless Object\textgreater addMisurazione(Misurazione mis)}
				\\Metodo che inserisce una nuova misurazione.
				\\Riceve in input un parametro formale di tipo Misurazioni e restituisce un Result di Object in caso di errore.

				\item \textbf {Result \textless Filtratore\textgreater getMisurazioni()}
				\\Metodo che raccoglie e memorizza i record relativi alle misurazioni presenti nel file \textit{ParametriClimatici (*.CSV)}.
				
				La complessit\`a del metodo \`e $\bigT$\textbf{(n)}.
			\end{itemize}

			\subsection{GestoreOperatore}
			La classe GestoreOperatore estende la classe Gestore, ne eredita tutti i metodi e contiene i seguenti:
			\begin{itemize}
				\item \textbf{Result\textless Operatore\textgreater registrazione(Operatore op, String pwd)}
				\\Metodo non implementato, in quanto l'operatore richiede un centro di monitoraggio che il GestoreCentro fallisce nel restituire, impedendo la corretta registrazione  dell'utente.
				
				La complessit\`a del metodo è \textbf{O(1)}.
				
				\item \textbf{Result \textless Operatore\textgreater login(String uid, String pwd)}
				\\Metodo che permette ad un operatore di effettuare il login.
				\\I parametri formali forniti in input sono lo userid e la password. In caso questi ultimi corrispondano ad un operatore esistente, il metodo permette all'operatore di effettuare l'accesso.
				\\In caso contrario restituisce una stringa relativa a un codice di errore.
				
				La complessit\`a del metodo \`e \textbf{O(n)}.
			\end{itemize}


		\section{Magazzeno}
		All'interno del package Magazzeno si trovano una serie di classi finalizzate a memorizzare informazioni oggetto id operazioni di lettura o scrittura su file.(*CSV).

			\subsection{AreaGeografica}
			La classe AreaGeografica implementa l'interfaccia DataTable, e contiene i seguenti metodi:
				\begin{itemize}
				\item \textbf{getter}
				\\getGeoID(), getLatitudine(), getLongitudine(), getStato(), getDenominazione().
				
				La complessit\`a del metodo è \textbf{O(1)}.
				
				\item \textbf{String toString()}
				\\Metodo che permette di stampare l'oggetto AreaGeografica, mostrando i campi denominazione, stato, latitudine e longitudine.
				
				La complessit\`a del metodo è \textbf{O(1)}.
				
				\item \textbf{boolean equals(Object obj)}
				\\Metodo che permette di confrontare un oggetto qualsiasi con un oggetto di tipo AreaGeografica.
				\\Prende in input come parametro formale un Object.
				\\Restituisce \textit{true} se l'oggetto di tipo Object \`e un istanza di AreaGeografica. In caso contrario restituisce \textit{false}.
				
				La complessit\`a del metodo è \textbf{O(1)}.
				
			\end{itemize}

			\subsection{CentroMonitoraggio}
			La classe CentroMonitoraggio implementa le due interfacce Convertable e Datatable e contiene i seguenti metodi:
			\begin{itemize}
				\item \textbf{getter}
				\\getNome(), getIndirizzo(), getAree().
				
				La complessit\`a dei metodi è \textbf{O(1)}.
				
				\item \textbf{String toString()}
				\\Metodo che permette di stampare l'oggetto CentroMonitoraggio con nome, indirizzo e aree associate.
				
				La complessit\`a del metodo è \textbf{O(1)}.
				
				\item \textbf{boolean equals(Object obj)}
				\\Metodo che permette di confrontare un oggetto qualsiasi con un oggetto di tipo CentroMonitoraggio. La funzione restituisce \textit{true} se l'oggetto \`e un' istanza di CentroMonitoraggio. In caso contrario restituisce \textit{false}.
				
				La complessit\`a del metodo è \textbf{O(1)}.
				
				\item \textbf{String toCSV()}
				\\Metodo che permette di creare una stringa che rappresenta CentroMonitoraggio nel formato (*.CSV) adoperato all'interno del programma.
				
				La complessit\`a del metodo è \textbf{O(1)}.
				
			\end{itemize}

			\subsection{DatoGeografico}
			La classe DatoGeografico implementa l'interfaccia DataTable e contiene i seguenti metodi:
			\begin{itemize}
				\item \textbf{void setDato(TipoDatoGeografico tipo, byte dato)}
				\\Metodo che imposta un dato geografico.
				\\Riceve come parametri il tipo di dato da impostare e il suo valore.
				
				La complessit\`a del metodo è \textbf{O(1)}.
				
				\item \textbf{byte getDato(TipoDatoGeografico tipo)}
				\\Metodo che riceve come parametro in input un oggetto TipoDatoGeografico e restituisce il tipo di dato geografico associato.
				
				La complessit\`a del metodo è \textbf{O(1)}.
				
				\item \textbf{String getNota(TipoDatoGeografico key)}
				\\Metodo che restituisce la nota del dato geografico che esegue il metodo.
				\\Restutuisce la nota associata alla chiave fornita come parametro.
				
				La complessit\`a del metodo è \textbf{O(1)}.
				
				\item \textbf{boolean setNota(TipoDatoGeografico key, String nota)}
				\\Metodo che permette di impostare le note relative al dato geografico.
				\\Riceve come parametri formali una chiave che stabilisce come inserire la nota all'interno del dato geografico e la nota da inserire.
				\\La funzione restituisce \textit{true} se l'operazione viene eseguita correttamente, \textit{false} altrimenti.
				
				La complessit\`a del metodo è \textbf{O(1)}.
				
				\item \textbf{boolean equals(Object obj)}
				\\Metodo che riceve in input un oggetto generico.
				\\Restituisce \textit{true} se l'oggetto che esegue il metodo \`e un'istanza di DatoGeografico, altrimenti \textit{false}.
				
				La complessit\`a del metodo è \textbf{O(1)}.
				
				\item \textbf{String toString()}
				\\Metodo che restituisce una stringa che rappresenta il dato geografico.
				
				La complessit\`a del metodo è \textbf{O(1)}.
				
				\item \textbf{boolean noteEquals(DatoGeografico dato)}
				\\Metodo che riceve in input un oggetto di tipo DatoGeografico da cui estrae la relativa nota e la confronta con quella che esegue il metodo.
				\\Restituisce \textit{true} se le due note sono uguali, \textit{false} altrimenti.
				
				La complessit\`a del metodo è \textbf{O(1)}.
				
				\item \textbf{boolean datoEquals(DatoGeografico dato)}
				\\Metodo che riceve in input come parametro formale un oggetto di tipo DatoGeografico.
				\\Restituisce \textit{true} se l'uguaglianza dei valori dati \`e verificata, \textit{false} altrimenti.
				
				La complessit\`a del metodo è \textbf{O(1)}.
				
			\end{itemize}

			\subsection{Filtratore}
			La classe Filtratore implementa le interfacce: Iterable, CercaAree, MediaAree e contiene i seguenti metodi:
			\begin{itemize}
				\item \textbf{Filtratore filtra(DataTable... dts)}
				\\Metodo che permette di filtrare le DataTable.
				\\Riceve come parametro una serie elementi di tipo Datatable.
				\\Restituisce le DataTable filtrate.
				
				La complessit\`a del metodo è \textbf{O(n*m)}.
				
				\item \textbf{Filtratore filtraOperatore(Operatore... ops)}
				\\Metodo che permette di filtrare gli operatori.
				\\Riceve come parametro formale una serie di elementi di tipo Operatore.
				\\Restituisce gli operatori filtrati.
				
				La complessit\`a del metodo è \textbf{O(n*m)}.
				
				\item \textbf{Filtratore filtraCentro(CentroMonitoraggio... cms)}
				\\Metodo che permette di filtrare i centri di monitoraggio.
				\\Riceve come parametro formale una serie di elementi di tipo CentroMonitoraggio.
				\\Restituisce i centri di monitoraggio filtrati.
				
				La complessit\`a del metodo è \textbf{O(n*m)}.
				
				\item \textbf{Filtratore filtraAree(AreaGeografica... ags)}
				\\Metodo che permette di filtrare le aree geografiche.
				\\Riceve come parametro formale una serie di elementi di tipo AreaGeografica.
				\\Restituisce le aree geografiche filtrate.
				
				La complessit\`a del metodo è \textbf{O(n*m)}.
				
				\item \textbf{Filtratore filtraNote(String... note)}
				\\Metodo che permette di filtrare le note.
				\\Riceve come parametro formale una serie di elementi di tipo String.
				\\Restituisce le note filtrate.
				
				La complessit\`a del metodo è \textbf{O(n*m)}.
				
				\item \textbf {Filtratore filtraDato(DatoGeografico... dati)}
				\\Metodo che permette di filtrare i dati geografici.
				\\Riceve come parametro formale una serie di elementi di tipo DatoGeografico.
				\\Restituisce i dati geografici filtrati.
				
				La complessit\`a del metodo è \textbf{O(n*m)}.
				
				\item \textbf {String toString()}
				\\Metodo che stampa l'oggetto filtratore.
				
				La complessit\`a del metodo \`e $\bigT$\textbf{(n)}.
				
				\item \textbf{DatoGeografico visualizzaAreaGeografica(AreaGeografica area)}
				\\Metodo che riceve in input come parametro un oggetto di tipo AreaGeografica.
				\\Conta quante volte appare un valore dell'Area Geografica.
				\\Restituisce un oggetto di tipo DatoGeografico e il valore associato.
				
				La complessit\`a del metodo \`e $\bigT$\textbf{(n)}.
				
				\item \textbf{Iterator<Misurazione> iterator()}
				
								
				\textbf{Metodo che restituisce un Iterator di Misurazione.	contenuto...}
				
				La complessit\`a del metodo è \textbf{O(1)}.
				
			\end{itemize}
			\subsection{Indirizzo}
			La classe Indirizzo definisce l'indirizzo dei centri di monitoraggio.
			\\Contiene i seguenti metodi:\\
			\begin{itemize}
				\item \textbf{getter}
				\\ getNomeVia(), getCivico(), getCap(), getComune(), getProvincia().
				
				La complessit\`a dei metodi è \textbf{O(1)}.
				
				\item \textbf{String toString()}
				\\Metodo che restituisce una stringa relativa all'indirizzo che lo esegue.
				
				La complessit\`a del metodo è \textbf{O(1)}.
				
				\item \textbf{String toCsv()}
				\\Metodo che permette di creare una stringa che descrive l'indirizzo nel formato (*.CSV) adoperato all'interno del programma.
				
				La complessit\`a del metodo è \textbf{O(1)}.
				
			\end{itemize}

			\subsection{ListaAree}
			La classe ListaAree implementa le interfacce Iterable, CercaAree e Convertable. Contiene i seguenti metodi:
			\begin{itemize}
				\item \textbf{boolean isEmpty()}
				\\Metodo che restituisce \textbf{true} se la ListaAree che esegue \`e vuota, \textit{false} altrimenti.
				
				La complessit\`a del metodo è \textbf{O(1)}.
				
				\item \textbf{AreaGeografica get(int k)}
				\\Metodo che riceve in input un intero k che indica la posizione di un AreaGeografica all'interno della ListaAree.
				\\Restituisce l'area geografica che \`e in posizione k.
				
				La complessit\`a del metodo è \textbf{O(n)}.
				
				\item \textbf{void add(AreaGeografica e, int k)}
				\\Metodo che aggiunge l'AreaGeografica in posizione k.
				\\Riceve in input un oggetto di tipo AreaGeografica e un intero k.
				
				La complessit\`a del metodo è \textbf{O(n)}.
				
				\item \textbf{AreaGeografica getFirst()}
				\\Metodo che restituisce il primo elemento presente nella ListaAree che esegue il metodo.
				
				La complessit\`a del metodo è \textbf{O(1)}.
				
				\item \textbf{AreaGeografica getLast()}
				\\Metodo che restituisce l'ultimo elemento presente nella ListaAree che esegue.
				
				La complessit\`a del metodo è \textbf{O(1)}.
				
				\item \textbf{int size()}
				\\Metodo che restituisce un intero che indica la dimensione della ListaAree che esegue.
				
				La complessit\`a del metodo \`e $\bigT$\textbf{(n)}.
				
				\item \textbf{void addFirst(AreaGeografica e)}
				\\Metodo che aggiunge in prima posizione della ListaAree che esegue un'area geografica, fornita in input come parametro.
				
				La complessit\`a del metodo è \textbf{O(1)}.
				
				\item \textbf{Iterator<AreaGeografica> iterator()}
				\\Metodo che restituisce un Iterator di AreaGeografica.
				
				La complessit\`a del metodo è \textbf{O(1)}.
				
				\item \textbf{ListaAree cercaAreaGeografica(String denominazione, String stato)}
				\\Metodo che ricerca le AreeGeografiche mediante i parametri denominazione e stato.
				\\Restituisce il risultato della ricerca in una ListaAree.
				
				La complessit\`a del metodo \`e $\bigT$\textbf{(n)}.
				
				\item \textbf{Result<AreaGeografica> cercaAreeGeografiche(double latitudine, double longitudine)}
				\\Metodo che ricerca le AreeGeografiche mediante i parametri forniti in input, latitudine e longitudine.
				\\Restituisce un Result di AreaGeografica.
				
				La complessit\`a del metodo è \textbf{O(n)}.
				
				\item \textbf{String toString()}
				\\Metodo che restituisce una stringa che rappresenta l'area geografica che esegue.
				
				La complessit\`a del metodo \`e $\bigT$\textbf{(n)}.
				
				\item \textbf{String toCsv()}
				\\Metodo che permette di creare una stringa per descrivere l'area geografica che esegue nel formato (*.CSV) adoperato all'interno del programma.
				
				La complessit\`a del metodo \`e $\bigT$\textbf{(n)}.
				
			\end{itemize}

			\subsection{Misurazioni}
			La classe Misurazioni implementa le interfacce Convertable e DataTable e contiene i seguenti metodi:\\
			\begin{itemize}
				\item \textbf{getter}
				\\getRid(), getDato(), getTime(), getTimeString(), getOperatore(), getCentro(), getArea().
				
				La complessit\`a dei metodi è \textbf{O(1)}.
				
				\item \textbf{String toString()}
				Metodo che restituisce una stringa che rappresenta la misurazione che esegue.
				
				La complessit\`a dei metodi è \textbf{O(1)}.
				
			\end{itemize}

			\subsection{Operatore}
			La classe Operatore definisce i seguenti metodi:\\
			\begin{itemize}
				\item \textbf{getter}
				\\getCf(), getCentro(), getCognome(), getNome(), getEmail(), getUid().
				
				La complessit\`a dei metodi è \textbf{O(1)}.
				
				\item \textbf{Result<Object> inserisciParametri(AreaGeografica area, DatoGeografico dato, LocalDateTime tempo)}
				\\Metodo che consente di inserire i dati climatici di una determinata area nel database i cui dati vengono forniti in input come parametri.
				\\Restituisce un Result di Object.
				
				La complessit\`a del metodo è \textbf{O(1)}.
				
				\item \textbf{String toString()}
				\\Metodo che restituisce una stringa che rappresenta l'operatore che esegue.
				
				La complessit\`a del metodo è \textbf{O(1)}.
				
				\item \textbf{String toCsv()}
				\\Metodo che crea una stringa relativa all'operatore che esegue nel formato (*.CSV) adoperato all'interno del programma.
				
				La complessit\`a del metodo è \textbf{O(1)}.
				
				\item \textbf{boolean equals(Object obj)}
				\\Metodo che confronta un oggetto qualsiasi con un oggetto di tipo Operatore.
				\\Riceve in input come parametro formale un Object.
				\\Restituisce \textit{true} se l'oggetto di tipo Object \`e un istanza di Operatore, \textit{false} altrimenti.
				
				La complessit\`a del metodo è \textbf{O(1)}.
				
			\end{itemize}

		\section{Utils}
			\subsection{listacustom}
			Package che contiene le classi che servono da supporto alla classe ListaAree.
				\subsubsection{CollezioniIterator}
				Classe che permette a ListaAree di svolgere l'istruzione "for-each loop". Implementa l'interfaccia Iterator.
				\begin{itemize}
					\item\textbf{E next()}
					\\Metodo che restituisce l'elemento corrente e scorre a quello successivo.
					
					La complessit\`a del metodo è \textbf{O(1)}.
					
					\item\textbf{boolean hasNext()}
					\\Metodo che restituisce \textit{true} se il nodo che esegue il metodo ha un successore, \textit{false} altrimenti.
					
					La complessit\`a del metodo è \textbf{O(1)}.
					
				\end{itemize}

				\subsubsection{Nodo}
				Classe che rappresenta i nodi della lista i cui elementi sono gestiti dai seguenti metodi:\\
				\begin{itemize}
					\item\textbf{setter}
					\\setDato(), setNext().
					
					La complessit\`a dei metodi è \textbf{O(1)}.
					
					\item\textbf{getter}
					\\getDato(), getNext().
					
					La complessit\`a dei metodi è \textbf{O(1)}.
					
				\end{itemize}

			\subsection{result}
				All'interno del package result sono presenti una serie di classi finalizzate a gestire i risultati di alcuni metodi dell'applicazione.
				
				\subsubsection{Panic}
				Classe che estende Error, finalizzata a gestire degli errori lanciati dalla classe Result che non \`e possibile catturare.

				\subsubsection{Result}
				Classe che si occupa della gestione dei risultati in alcuni metodi che potrebbero lanciare errori nell'applicazione.\\
				\begin{itemize}
					\item \textbf{getter}
					\\getError(), getMessage(), getFullMessage().
					
					La complessit\`a dei metodi è \textbf{O(1)}.
					
					\item\textbf{boolean isValid()}
					\\Metodo che restituisce \textit{true} se il Result \`e valido, \textit{false} altrimenti.
					
					La complessit\`a del metodo è \textbf{O(1)}.
					
					\item\textbf{boolen isError()}
					\\Metodo che restituisce \textit{true} se il Result lancia un errore, \textit{false} altrimenti.
					
					La complessit\`a del metodo è \textbf{O(1)}.
	
					\item\textbf{void ifValid(BiConsumer\textless T, Integer \textgreater fn)}
					\\Metodo che esegue la funzione data come parametro se il Result \`e valido.
					
					La complessit\`a del metodo è \textbf{O(1)}.
	
					\item\textbf{void ifError(BiConsumer\textless T, Integer\textgreater fn)}
					\\Metodo che esegue la funzione data come parametro se il Result genera errore.
					
					La complessit\`a del metodo è \textbf{O(1)}.
	
					\item \textbf{T get()}
					\\Metodo che restituisce il contenuto di Result.
					
					La complessit\`a del metodo è \textbf{O(1)}.
	
					\item \textbf{T getOr(T other)}
					\\Metodo che restituisce il contenuto di Result se questo non \`e nullo. Altrimenti restituisce il parametro other.
					
					La complessit\`a del metodo è \textbf{O(1)}.
	
					\item \textbf{T getOrElse(Supplier \textless T\textgreater fn)}
					\\Metodo che restituisce il contenuto di Result se questo non \`e nullo. In caso contrario esegue la funzione fornita come parametro e restituisce il risultato di quest'ultima.
					
					La complessit\`a del metodo è \textbf{O(1)}.
	
					\item  \textbf{T except()}
					\\Metodo che restituisce il contenuto di Result senza eseguire nessun controllo.
					
					La complessit\`a del metodo è \textbf{O(1)}.
	
					\item \textbf{void panic()}
					\\Metodo che lancia un errore non catturabile.
					
					La complessit\`a del metodo è \textbf{O(1)}.
					
				\end{itemize}

			\subsection{terminal}
				\subsubsection{Screen}
				Classe che contiene il seguente metodo:
				\begin{itemize}
					\item \textbf{void show(View v)}
					\\Metodo finalizzato a mostrare View. Pulisce il terminale prima e dopo l'esecuzione dell'applicazione.
					
					La complessit\`a del metodo è \textbf{O(1)}.
					
				\end{itemize}

				\subsubsection{Terminal}
				Classe Involucro che racchiude System.in() e System.out(), aggiungendo varie funzionalit\`a.
				\begin{itemize}
					\item \textbf{void clear()}
					\\Metodo che pulisce la console con il codice di uscita ANSI.
					
					La complessit\`a del metodo è \textbf{O(1)}.
					
					\item \textbf{void printf(String str, Object... args)}
					\\Metodo che formatta e stampa una stringa nel terminale.
					\\I parametri formali servono a stampare all'utente una stringa (str) interpolata (args).
					
					La complessit\`a del metodo è \textbf{O(1)}.
					
					\item \textbf{void printfln(String str, Object... args)}
					\\Metodo che formatta e stampa una stringa nel terminale.
					\\I parametri formali sono finalizzati a stampare all'utente una stringa (str) interpolata (args).
					
					La complessit\`a del metodo è \textbf{O(1)}.
					
					\item \textbf{String readLine()}
					\\Metodo che permette di leggere una linea dalla console utente.
					\\Restituisce la stringa inserita dall'utente.
					
					La complessit\`a del metodo è \textbf{O(1)}.
					
					\item \textbf{String readLine(String str, Object... args)}
					\\Metodo che stampa una stringa e aspetta una risposta dell'utente.
					\\I parametri formali sono finalizzati a stampare all'utente una stringa (str) interpolata (args).
					
					La complessit\`a del metodo è \textbf{O(1)}.
					
					\item \textbf{String readLineOrDefault(String def, String str, Object... args)}
					\\Metodo che stampa una stringa all'utente e aspetta una risposta, se quest'ultima non viene data (stringa vuota), il metodo restituisce (def).
					\\I parametri formali sono finalizzati a stampare all'utente una stringa (str) interpolata (args).
					
					La complessit\`a del metodo è \textbf{O(1)}.
					
					\item \textbf{String readWhile(Predicate<String> fn, String str, Object... args)}
					\\Stampa una stringa e aspetta una risposta dall'utente, controllando la stringa fornita come parametro.
					\\I parametri formali sono finalizzati a stampare all'utente una stringa (str) interpolata (args). In particolare fn \`e una funzione che restituisce un booleano.
					
					La complessit\`a del metodo è \textbf{O(n)}.
	
					\item \textbf{boolean promptUser(boolean yes, String str, Object... args)}
					\\Metodo che pone all'utente una domanda con risposta di tipo si/no.
					\\I parametri formali sono finalizzati a stampare all'utente una stringa (str) interpolata (args). In particolare rappresenta la scelta di default.
					
					La complessit\`a del metodo è \textbf{O(1)}.
					
				\end{itemize}

				\subsubsection{View}
				Interfaccia che contiene il seguente metodo:
				\begin{itemize}
					\item \textbf{abstract void start(Terminal term)}
					\\Metodo astratto finalizzato a avviare una schermata dell'applicazione attraverso il terminale.
					
					La complessit\`a del metodo è \textbf{O(1)}.
				\end{itemize}

			\subsection{CercaAree}
			L'implementazione di questa interfaccia consente la ricerca di aree geografiche.
			\begin{itemize}
				\item \textbf{ListaAree cercaAreaGeografica (String denominazione, String stato)}
				\\Metodo che ricerca delle aree geografiche mediante denominazione e stato di appartenenza.
				\\Restituisce un Result di tipo AreaGeografica contenente le aree geografiche corrispondenti al risultato della ricerca.
				
				\item\textbf{Result\textless AreaGeografica\textgreater cercaAreeGeografiche(double latitudine, double longitudine)}
				\\Metodo che ricerca delle aree geografiche mediante coordinate (latitudine e longitudine fornite in input come parametro).
				\\Restituisce un Result di tipo AreaGeografica contenente le aree geografiche corrispondenti al risultato della ricerca.
			\end{itemize}

			\subsection{Convertable}
			L'implementazione di questa interfaccia consente a un oggetto di essere convertito nel formato (*.CSV).
			\begin{itemize}
				\item \textbf{String toCsv()}
				\\Metodo che converte l'oggetto nel formato (*.CSV).
				\\Restituisce una stringa.
			\end{itemize}
			\subsection{DataTable}
			L'implementazione di questa interfaccia permette di confrontare due record.
			\begin{itemize}
				\item\textbf{boolean equals(Object obj)}
				\\Metodo che confronta l'oggetto che esegue con l'oggetto fornito come parametro.
				\\Restituisce \textit{true} se sono uguali, \textit{false} altrimenti.
			\end{itemize}
			\subsection{MediaAree}
			La classe che implementa questa interfaccia permette di visualizzare le informazioni relative ad un'area geografica.
			\begin{itemize}
				\item\textbf {DatoGeografico visualizzaAreaGeografica (AreaGeografica area)}
				\\Metodo che restituisce un nuovo dato geografico, che rappresenta un prospetto riassuntivo dei parametri climatici associati all'area geografica fornita in input.
			\end{itemize}
			\subsection{DatoGeografico}
			Enumerativo che rappresenta il tipo di un dato geografico.


		\section{Main}

	\nocite{IuriTex}
	\bibliographystyle{alpha}
	\bibliography{bib/biblio}
	\printindex

\end{document}



