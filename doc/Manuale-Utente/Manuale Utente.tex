\documentclass[a4paper,12pt,twoside]{article}
%questa e la document class, puoi modificare la dimensione dei caratteri, del foglio e altro

\usepackage{hyperref}
%\usepackage{showkeys} mostra i label che scegli

%package xypic
\usepackage{xypic}
\usepackage{caption}

\usepackage{url}

%se ti serve un pacchetto aggiungili qui sopra

\RequirePackage{booktabs, multirow}
%%%%
%%%%%%%

\title {
	CLIMATE MONITORING \\
	{Manuale utente}}

%Titolo e nelle parentesi quadre abbreviazione

\author{
	Iuri Antico \textit{matricola}:
	\texttt{753144}
	\and \\
	Michael Bernasconi \textit{matricola}:
	\texttt{752259}
	\and \\
	Gabriele Borgia \textit{matricola}:
	\texttt{753262}
	\and \\
	Beatrice Balzarini \textit{matricola}:
	\texttt{752257}
}
	
\date{\today}

\begin{document}
	
	\makeatletter
	\begin{titlepage}
		\maketitle
	\end{titlepage}
	\makeatother

	\tableofcontents

	\newpage
	
	\section{Introduzione}
	Climate Monitoring è un'applicazione di monitoraggio di parametri climatici fornita da centri di monitoraggio sul territorio italiano, in grado di rendere disponibili, a operatori ambientali e comuni cittadini, i dati relativi alla propria zona di interesse. \\	
	In particolare, un cittadino ha la possibilità di visualizzare i parametri climatici di proprio interesse, mentre un operatore può gestirli in base alle necessità del centro di monitoraggio di cui si occupa.
	
		\subsection{Funzionamento generale dell'applicazione}
		L'applicazione usa i dati forniti dal file CoordinateMonitoraggio:
	
	\newpage
			
	\section{Avviare l'applicazione}
	
		\subsection{Requisiti minimi}
		
		\subsection{Avviare l'applicazione}
		
		\subsection{Troubleshooting}
		
	\newpage
		
	\section{Schermata principale}
	
	\newpage
	
	\section{Menù principale}
	\subsection{Voci disponibili}
	\subsection{Funzionamento}
			\subsubsection{Esci}
			
	\newpage
	
	\section{Cerca le Misurazioni}
	\section{Login}
	\section{Registrazione}

	\section{Limiti della soluzione sviluppata}
	\section{Sitografia/ Bibliografia}
	

		
	
	

	

	\bibliographystyle{amsalpha}

	
	
	
	
	
\end{document}



