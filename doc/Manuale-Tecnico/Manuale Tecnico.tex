\documentclass[a4paper, 12pt]{report}

\usepackage[margin=3cm]{geometry}
\usepackage{hyperref}
%\usepackage{showkeys} % mostra i label che scegli
\usepackage{graphicx}
\usepackage{parskip}
\usepackage{caption}
\usepackage{url}

% Mathematics
\usepackage{amsmath}
\usepackage{amssymb}

\newcommand{\bbN}{\mathbb{N}}
\newcommand{\bbZ}{\mathbb{Z}}
\newcommand{\bbQ}{\mathbb{Q}}
\newcommand{\bbR}{\mathbb{R}}
\newcommand{\bbC}{\mathbb{C}}

\newcommand{\bigO}{\mathbf{O}}
\newcommand{\litO}{\mathbf{o}}
\newcommand{\bigM}{\mathbf{\Omega}}
\newcommand{\bigT}{\mathbf{\Theta}}

% Additional options
\setcounter{tocdepth}{3}
\setcounter{secnumdepth}{5}

\title {
	CLIMATE MONITORING \\
	Manuale Tecnico}

% Titolo e nelle parentesi quadre abbreviazione

\author{
	Iuri Antico \textit{matricola}:
	\texttt{753144}
	\and \\
	Michael Bernasconi \textit{matricola}:
	\texttt{752259}
	\and \\
	Gabriele Borgia \textit{matricola}:
	\texttt{753262}
	\and \\
	Beatrice Balzarini \textit{matricola}:
	\texttt{752257}
}

\date{\today}

\begin{document}

	\makeatletter
	\begin{titlepage}
		\maketitle
	\end{titlepage}
	\makeatother

	\tableofcontents
	\listoffigures
	\listoftables

	\chapter{Introduzione}
		\section{Librerie esterne utilizzate}
			\subsection{Apache Commons CSV}
			E' una libreria di Apache che fornisce i metodi per la gestione dei file (*.CSV). In particolare è stata usata nel progetto per la lettura e scrittura dei dati da memorizzare.

	\chapter{Struttura generale del sistema di classi}
		\section{cli}
			\subsection{Help}
			\subsection{Login}
			\subsection{MainMenu}
			\subsection{MostraMisurazioni}
			\subsection{Registrazione}

		\section{gestori}
		All'interno del package Gestori si trovano una serie di classi finalizzate alle operazioni di lettura e scrittura su File contenenti dati utili al monitoraggio di parametri climatici sul territorio italiano.
		Ad ogni record memorizzato su file viene associato un indice (ID) univoco.
			\subsection{DataBase}
			La classe DataBase si occupa di creare (per ogni classe che estende Gestore) un oggetto in grado di richiamare le funzioni associate.
			\subsection{Gestore}
			La classe astratta Gestore contiene metodi relativi la gestione dei file contenenti i dati d'interesse. In particolare:
			\begin{itemize}
				\item \textbf{close()}
				Metodo che si occupa di aprire un file.
				\item \textbf{reload()}
				Metodo che si occupa di ricaricare (chiudere e riaprire un file)
				\item  \textbf{setProperty(String s, String k)}
				Metodo che si occupa di impostare una proprietà nel file (*.CSV.DAT) associato ad una tabella (*.CSV)
			\end{itemize}

			\subsection{GestoreArea}
			la classe GestoreArea estende la classe Gestore e ne eredita tutti i metodi
			implementa l'interfaccia CercaAree e contiene i seguenti metodi:
			\begin{itemize}
				\item \textbf{getArea()}
				Metodo che ricerca una determinata area geografica in base al suo ID
			\end{itemize}

			\subsection{GestoreCentro}
			la classe GestoreCentro estende la classe Gestore, eredita tutti i metodi e contiene i seguenti:
			\begin{itemize}
				\item \textbf{getCentro()}
				Metodo che ricerca un determinato centro di monitoraggio in base al nome
				\item \textbf {addCentro ()}
				Metodo che crea un nuovo record relativo a un determinato centro di monitoraggio e lo memorizza nel file CentriMonitoraggio.CSV
			\end{itemize}

			\subsection{GestoreDato}
			la classe GestoreDato estende la classe Gestore, eredita tutti i metodi e contiene i seguenti:
			\begin{itemize}
				\item \textbf{getDato()}
				Metodo che ricerca un determinato dato geografico in base al suo ID
				\item \textbf {addDato()}
				Metodo che crea un nuovo record relativo a un determinato dato geografico e lo memorizza nel file ParametriClimatici.CSV
			\end{itemize}

			\subsection{GestoreMisurazioni}
			la classe GestoMisurazioni estende la classe Gestore, eredita tutti i metodi e contiene i seguenti:
			\begin{itemize}
				\item \textbf{addMisurazione()}
				Metodo che crea un nuovo record relativo a una determinata misurazione
				\item \textbf {getMisurazioni()}
				Metodo che memorizza i record relativi alle misurazioni presenti nel file ParametriClimatici.CSV in una lista, la quale inizializza un nuovo oggetto di tipo Filtratore.
			\end{itemize}

			\subsection{GestoreOperatore}
			la classe GestoreOperatore estende la classe Gestore, eredita tutti i metodi e contiene i seguenti:
			\begin{itemize}
				\item \textbf{registrazione()}
				Metodo che permette a un operatore di registrarsi impostando una password
				\item \textbf{login()}
				Metodo che permette a un utente di effettuare il login
				\item \textbf{getOperatore()}
				Metodo che crea un nuovo record relativo a un determinato operatore e lo memorizza nel file OperatoriRegistrati.CSV
			\end{itemize}

		\section{Magazzeno}
		All'interno del package Magazzeno si trovano una serie di classi che servono per memorizzare informazioni che vengono lette o che dovranno essere scritte su file.(*CSV).

			\subsection{AreaGeografica}
			la classe AreaGeografica implementa l'interfaccia DataTable, che definisce tutti i metodi e contiene i seguenti:
			I metodi getter sono: getGeoID(), getLatitudine(), getLongitudine(), getStato(), getDenominazione().
			\begin{itemize}
				\item \textbf{toString()}
				Metodo che permette di stampare l'oggetto AreaGeografica, con i campi denominazione, stato, latitudine e longitudine.
				\item \textbf{equals()}
				Metodo che permette di confrontare un oggetto qualsiasi con un oggetto di tipo AreaGeografia.
			\end{itemize}

			\subsection{CentroMonitoraggio}
			la classe CentroMonitoraggio implementa due interfacce: Convertable e Datatable che definisce tutti i metodi e contiene i seguenti:
			i metodi getter sono: getNome(), getIndirizzo(), getAree().
			\begin{itemize}
				\item \textbf{toString()}
				Metodo che permette di stampare l'oggetto CentroMonitoraggio con i campi nome, indirizzo e le aree associate ad esso.
				\item \textbf{equals()}
				Metodo che permette di confrontare un oggetto qualsiasi con un oggetto di tipo CentroMonitoraggio.
				\item \textbf{toCSV()}
				Metodo che permette di creare una stringa nel formato (*.CSV) adoperato all'interno del programma.
			\end{itemize}

			\subsection{DatoGeografico}
			La classe DatoGeografico implementa l'interfaccia DataTable che definisce tutti i metodi e contiene i seguenti:
%
			\subsection{Filtratore}
			\subsection{Indirizzo}
			\subsection{ListaAree}
			\subsection{Misurazioni}
			\subsection{Operatore}

		\section{Utils}
			\subsection{listacustom}
				\subsubsection{CollezioniIterator}
			\subsection{listacustom}
				\subsubsection{Nodo}
			\subsection{result}
				\subsubsection{Panic}
				\subsubsection{Result}
				\subsubsection{Either}

			\subsection{terminal}
				\subsubsection{Screen}
				\subsubsection{Terminal}
				\subsubsection{View}

			\subsection{CercaAree}
			\subsection{Convertable}
			\subsection{DataTable}
			\subsection{MediaAree}
			\subsection{DatoGeografico}

		\section{Main}

	\bibliographystyle{report}

\end{document}



