\documentclass[a4paper, 12pt,twoside]{article}
%questa e la document class, puoi modificare la dimensione dei caratteri, del foglio e altro

\usepackage{hyperref}
%\usepackage{showkeys} mostra i label che scegli

%package xypic
\usepackage{xypic}
\usepackage[all,cmtip]{xy}
\usepackage{tikz-cd}
\usepackage{caption}

\usepackage{url}

%se ti serve un pacchetto aggiungili qui sopra

\RequirePackage{booktabs, multirow}
\RequirePackage{pgf}
%%%%
%%%%%%%

\title {
	CLIMATE MONITORING \\
	{Manuale utente}}

%Titolo e nelle parentesi quadre abbreviazione

\author{
	Iuri Antico \textit{matricola}:
	\texttt{753144}
	\and \\
	Michael Bernasconi \textit{matricola}:
	\texttt{752259}
	\and \\
	Gabriele Borgia \textit{matricola}:
	\texttt{753262}
	\and \\
	Beatrice Balzarini \textit{matricola}:
	\texttt{752257}
}
	
\date{\today}

\begin{document}
	
	\makeatletter
	\begin{titlepage}
		\maketitle
	\end{titlepage}
	\makeatother

	\tableofcontents

	\newpage
	
	\section{Introduzione}
		\subsection{Librerie esterne utilizzate}
		
			\subsubsection{Apache Commons CSV}
			
	\section{Struttura generale del sistema di classi}
		\subsection{Cli}
			\subsubsection{Help}
			\subsubsection{Login}
			\subsubsection{MainMenu}
			\subsubsection{MostraMisurazioni}
			\subsubsection{Registrazione}

		
		
		\subsection{Gestori}
		All'interno del package Gestori si trovano una serie di classi finalizzate alle operazioni di lettura e scrittura su File contenenti dati utili al monitoraggio di parametri climatici sul territorio italiano.
		Ad ogni record memorizzato su file viene associato un indice (ID) univoco.
				
			\subsubsection{DataBase}
			La classe DataBase si occupa di creare (per ogni classe che estende Gestore) un oggetto in grado di richiamare le funzioni associate.
				
			\subsubsection{Gestore}
			La classe astratta Gestore contiene metodi relativi la gestione dei file contenenti i dati d'interesse. In particolare:
			\begin{itemize}
				\item \textbf{start() e close ()}
				Metodi che si occupano rispettivamente di aprire e chiudere un file.
				\item \textbf{reload()}
				Metodo che si occupa di ricaricare (chiudere e riaprire un file)
				\item  \textbf{setProperty()}
				Metodo che si occupa di impostare una proprietà nel file (*.CSV.DAT) associato ad una tabella (*.CSV)
			\end{itemize}
			
			\newpage
				
			
				
			\subsubsection{GestoreArea}
			la classe GestoreArea estende la classe Gestore e ne eredita tutti i metodi
			implementa l'interfaccia CercaAree e contiene i seguenti metodi:
			\begin{itemize}
				\item \textbf{getArea()}
				Metodo che ricerca una determinata area geografica in base al suo ID
			\end{itemize}
			
			\subsubsection{GestoreCentro}
			la classe GestoreCentro estende la classe Gestore, eredita tutti i metodi e contiene i seguenti:
			\begin{itemize}
				\item \textbf{getCentro()}
				Metodo che ricerca un determinato centro di monitoraggio in base al nome
				\item \textbf {addCentro ()}
				Metodo che crea un nuovo record relativo a un determinato centro di monitoraggio e lo memorizza nel file CentriMonitoraggio.CSV
			\end{itemize}
			
			\subsubsection{GestoreDato}
			la classe GestoreDato estende la classe Gestore, eredita tutti i metodi e contiene i seguenti:
			\begin{itemize}
				\item \textbf{getDato()}
				Metodo che ricerca un determinato dato geografico in base al suo ID
				\item \textbf {addDato()}
				Metodo che crea un nuovo record relativo a un determinato dato geografico e lo memorizza nel file ParametriClimatici.CSV
			\end{itemize}
			
			\subsubsection{GestoreMisurazioni}
			la classe GestoMisurazioni estende la classe Gestore, eredita tutti i metodi e contiene i seguenti:
			\begin{itemize}
				\item \textbf{addMisurazione()}
				Metodo che crea un nuovo record relativo a una determinata misurazione
				\item \textbf {getMisurazioni()}
				Metodo che memorizza i record relativi alle misurazioni presenti nel file ParametriClimatici.CSV in una lista, la quale inizializza un nuovo oggetto di tipo Filtratore.
			\end{itemize}
			
			
			\newpage
			\subsubsection{GestoreOperatore}
			la classe GestoreOperatore estende la classe Gestore, eredita tutti i metodi e contiene i seguenti:
			\begin{itemize}
				\item \textbf{registrazione()}
				Metodo che permette a un operatore di registrarsi impostando una password
				\item \textbf{login()}
				Metodo che permette a un utente di effettuare il login
				\item \textbf{getOperatore()}
				Metodo che crea un nuovo record relativo a un determinato operatore e lo memorizza nel file OperatoriRegistrati.CSV
			\end{itemize}
			
			\newpage
			
		\subsection{Magazzeno}
			\subsubsection{AreaGeografica}
			\subsubsection{CentroMonitoraggio}
			\subsubsection{DatoGeografico}
			\subsubsection{Filtratore}
			\subsubsection{Indirizzo}
			\subsubsection{ListaAree}
			\subsubsection{Misurazioni}
			\subsubsection{Operatore}
		
		\subsection{Utils}
			\subsubsection{listacustom.CollezioniIterator}
			\subsection{listacustom.Nodo}
			\subsection{result.Blunder}
			\subsection{result.Outcome}
			\subsection{result.Panic}
			\subsection{result.Result}
			
			\subsection{terminal.Screen}
			\subsection{terminal.Terminal}
			\subsection{terminal.View}
			
			\subsection{CercaAree}
			\subsection{Convertable}
			\subsection{DataTable}
			\subsection{MediaAree}
			\subsection{DatoGeografico}
		
		
		\subsection{Main}
		\newpage

	
	

		
	
	

	

	\bibliographystyle{amsalpha}

	
	
	
	
	
\end{document}



